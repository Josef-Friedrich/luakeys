\documentclass{ltxdoc}

\usepackage[
  colorlinks=true,
  linkcolor=red,
  filecolor=red,
  urlcolor=red,
]{hyperref}
\EnableCrossrefs
\CodelineIndex
\RecordChanges

\usepackage{mdframed}
\usepackage{minted}
\usepackage{luakeys-debug}
\usepackage{multicol}
\usemintedstyle{friendly}
\BeforeBeginEnvironment{minted}{\begin{mdframed}}
\AfterEndEnvironment{minted}{\end{mdframed}}
\setminted{
  breaklines=true,
  fontsize=\footnotesize,
}
\def\lua#1{\mintinline{lua}|#1|}

\begin{document}

\providecommand*{\url}{\texttt}

\title{The \textsf{luakeys} package}
\author{%
  Josef Friedrich\\%
  \url{josef@friedrich.rocks}\\%
  \href{https://github.com/Josef-Friedrich/nodetree}{github.com/Josef-Friedrich/luakeys}%
}
\date{v0.1 from 2021/01/10}

\maketitle

\vfill

% \luakeysdebug{level1={level2={level3={dim=1cm,bool=true,num=-1e-03,str=lua}}}}

\begin{minted}{lua}
local luakeys = require('luakeys')
local kv = luakeys.parse('level1={level2={level3={dim=1cm,bool=true,num=-1e-03,str=lua}}}')
luakeys.print(kv)
\end{minted}

\noindent
Result:

\begin{center}
\begin{minted}{lua}
{
  ['level1'] = {
    ['level2'] = {
      ['level3'] = {
        ['dim'] = 1864679,
        ['bool'] = true,
        ['num'] = -0.001
        ['str'] = 'lua',
      }
    }
  }
}
\end{minted}
\end{center}

\vfill

\strut

\newpage

\tableofcontents

\newpage

\noindent
|luakeys| is a Lua module that can parse key-value options like the
\TeX{} packages \href{https://www.ctan.org/pkg/keyval}{keyval},
\href{https://www.ctan.org/pkg/kvsetkeys}{kvsetkeys},
\href{https://www.ctan.org/pkg/kvoptions}{kvoptions},
\href{https://www.ctan.org/pkg/xkeyval}{xkeyval},
\href{https://www.ctan.org/pkg/pgfkeys}{pgfkeys} etc. But |luakeys|
accompilshes this task entirely using the Lua language and doesn’t rely
on \TeX{} macros. Therefore this package can only be used with the
\TeX{} engine Lua\TeX{}. Since |luakeys| uses
\href{http://www.inf.puc-rio.br/~roberto/lpeg/}{LPeg}, the parsing
mechanism should be pretty robust.

%-----------------------------------------------------------------------
% Index
%-----------------------------------------------------------------------

\section{Recognized data types}

%-----------------------------------------------------------------------
%
%-----------------------------------------------------------------------

\subsection{boolean}

\begin{multicols}{2}
\begin{minted}{latex}
\luakeysdebug{
  lower case true = true,
  upper case true = TRUE,
  title case true = True
  lower case false = false,
  upper case false = FALSE,
  title case false = False,
}
\end{minted}
\begin{minted}{lua}
{
  ['lower case true'] = true,
  ['upper case true'] = true,
  ['title case true'] = true,
  ['lower case false'] = false,
  ['upper case false'] = false
  ['title case false'] = false,
}
\end{minted}
\end{multicols}

%-----------------------------------------------------------------------
%
%-----------------------------------------------------------------------

\subsection{number}

\begin{multicols}{2}
\begin{minted}{latex}
\luakeysdebug{
  num1 = 4,
  num2 = 4,
  num3 = 0.4,
  num4 = 4.57e-3,
  num5 = 0.3e12,
  num6 = 5e+20
}
\end{minted}
\begin{minted}{lua}
{
  ['num1'] = 4,
  ['num2'] = 4,
  ['num3'] = 0.4,
  ['num4'] = 0.00457,
  ['num5'] = 300000000000.0,
  ['num6'] = 5e+20
}
\end{minted}
\end{multicols}

%-----------------------------------------------------------------------
%
%-----------------------------------------------------------------------

\subsection{dimension}

\begin{multicols}{2}
\begin{minted}{latex}
\luakeysdebug{
  bp = 1bp,
  cc = 1cc,
  cm = 1cm,
  dd = 1dd,
  em = 1em,
  ex = 1ex,
  in = 1in,
  mm = 1mm,
  nc = 1nc,
  nd = 1nd,
  pc = 1pc,
  pt = 1pt,
  sp = 1sp,
}
\end{minted}
\begin{minted}{lua}
{
  ['bp'] = 65781,
  ['cc'] = 841489,
  ['cm'] = 1864679,
  ['dd'] = 70124,
  ['em'] = 655360,
  ['ex'] = 282460,
  ['in'] = 4736286,
  ['mm'] = 186467,
  ['nc'] = 839105,
  ['nd'] = 69925,
  ['pc'] = 786432,
  ['pt'] = 65536,
  ['sp'] = 1,
}
\end{minted}
\end{multicols}

%-----------------------------------------------------------------------
%
%-----------------------------------------------------------------------

\subsection{string}

\luakeysdebug{
  string1 = without quotes,
  string3 = "with double quotes",
}

\begin{multicols}{2}
\begin{minted}{latex}
\luakeysdebug{
  string1 = without quotes,
  %string2 = 'with single quotes',
  string3 = "with double quotes",
}
\end{minted}
\begin{minted}{lua}
{
  ['bp'] = 65781,
  ['cc'] = 841489,
  ['cm'] = 1864679,
  ['dd'] = 70124,
  ['em'] = 655360,
  ['ex'] = 282460,
  ['in'] = 4736286,
  ['mm'] = 186467,
  ['nc'] = 839105,
  ['nd'] = 69925,
  ['pc'] = 786432,
  ['pt'] = 65536,
  ['sp'] = 1,
}
\end{minted}
\end{multicols}

\section{Exported functions of the luakeys.lua module}

\subsection{\lua{luakeys.parse(input_string, options)}}

The function \lua{parse(input_string, options)} is the main method of
this module.

\begin{minted}{latex}
\newcommand{\mykeyvalcmd}[1][]{
  \directlua{
    result = luakeys.parse('#1')
    luakeys.print(result)
  }
  #2
}
\mykeyvalcmd[one=1]{test}
\end{minted}

In plain \TeX:

\begin{minted}{latex}
\def\mykeyvalcommand#1{
  \directlua{
    result = luakeys.parse('#1')
    luakeys.print(result)
  }
}
\mykeyvalcmd{one=1}
\end{minted}

\subsection{\lua{luakeys.print(lua_table)}}

\subsection{\lua{luakeys.stringify(lua_table, for_tex)}}

\subsection{\lua{luakeys.render(lua_table)}}

The function \lua{render(lua_table)} works in reverse of the function
\lua{parse(lua_table)}. It takes a Lua table and converts this table
into a key-value string. The resulting string usually has a different
order as your input table.

\begin{minted}{lua}
result = luakeys.parse('one=1,two=2,tree=3,')
print(luakeys.render(result))
--- one=1,two=2,tree=3,
--- or:
--- two=2,one=1,tree=3,
--- or:
--- ...
\end{minted}

In Lua, only tables with 1-based consecutive integer keys (a.k.a. array
tables) can be parsed in order.

\begin{minted}{lua}
result = luakeys.parse('one,two,three')
print(luakeys.render(result))
--- one,two,three, (always)
\end{minted}

\pagebreak
\PrintChanges
\pagebreak
\PrintIndex
\end{document}
